\documentclass[12pt,twoside]{article}

\newcommand{\reporttitle}{477 - Computational Optimisation}
\newcommand{\reportauthor}{Jiahao Lin}
\newcommand{\reporttype}{Coursework 2}
\newcommand{\cid}{00837321}

% include files that load packages and define macros
%%%%%%%%%%%%%%%%%%%%%%%%%%%%%%%%%%%%%%%%%
% University Assignment Title Page 
% LaTeX Template
% Version 1.0 (27/12/12)
%
% This template has been downloaded from:
% http://www.LaTeXTemplates.com
%
% Original author:
% WikiBooks (http://en.wikibooks.org/wiki/LaTeX/Title_Creation)
%
% License:
% CC BY-NC-SA 3.0 (http://creativecommons.org/licenses/by-nc-sa/3.0/)
% 
% Instructions for using this template:
% This title page is capable of being compiled as is. This is not useful for 
% including it in another document. To do this, you have two options: 
%
% 1) Copy/paste everything between \begin{document} and \end{document} 
% starting at \begin{titlepage} and paste this into another LaTeX file where you 
% want your title page.
% OR
% 2) Remove everything outside the \begin{titlepage} and \end{titlepage} and 
% move this file to the same directory as the LaTeX file you wish to add it to. 
% Then add \input{./title_page_1.tex} to your LaTeX file where you want your
% title page.
%
%----------------------------------------------------------------------------------------
%	PACKAGES AND OTHER DOCUMENT CONFIGURATIONS
%----------------------------------------------------------------------------------------
\usepackage{ifxetex}
\usepackage{textpos}
\usepackage{natbib}
\usepackage{kpfonts}
\usepackage[a4paper,hmargin=2.8cm,vmargin=2.0cm,includeheadfoot]{geometry}
\usepackage{ifxetex}
\usepackage{stackengine}
\usepackage{tabularx,longtable,multirow,subfigure,caption}%hangcaption
\usepackage{fncylab} %formatting of labels
\usepackage{fancyhdr}
\usepackage{color}
\usepackage[tight,ugly]{units}
\usepackage{url}
\usepackage{float}
\usepackage[english]{babel}
\usepackage{amsmath}
\usepackage{graphicx}
\usepackage[colorinlistoftodos]{todonotes}
\usepackage{dsfont}
\usepackage{epstopdf} % automatically replace .eps with .pdf in graphics
\usepackage{natbib}
\usepackage{backref}
\usepackage{array}
\usepackage{latexsym}
\usepackage{etoolbox}

\usepackage{enumerate} % for numbering with [a)] format 



\ifxetex
\usepackage{fontspec}
\setmainfont[Scale=.8]{OpenDyslexic-Regular}
\else
\usepackage[pdftex,pagebackref,hypertexnames=false,colorlinks]{hyperref} % provide links in pdf
\hypersetup{pdftitle={},
  pdfsubject={}, 
  pdfauthor={\reportauthor},
  pdfkeywords={}, 
  pdfstartview=FitH,
  pdfpagemode={UseOutlines},% None, FullScreen, UseOutlines
  bookmarksnumbered=true, bookmarksopen=true, colorlinks,
    citecolor=black,%
    filecolor=black,%
    linkcolor=black,%
    urlcolor=black}
\usepackage[all]{hypcap}
\fi

\usepackage{tcolorbox}

% various theorems
\usepackage{ntheorem}
\theoremstyle{break}
\newtheorem{lemma}{Lemma}
\newtheorem{theorem}{Theorem}
\newtheorem{remark}{Remark}
\newtheorem{definition}{Definition}
\newtheorem{proof}{Proof}

% example-environment
\newenvironment{example}[1][]
{ 
\vspace{4mm}
\noindent\makebox[\linewidth]{\rule{\hsize}{1.5pt}}
\textbf{Example #1}\\
}
{ 
\noindent\newline\makebox[\linewidth]{\rule{\hsize}{1.0pt}}
}



%\renewcommand{\rmdefault}{pplx} % Palatino
% \renewcommand{\rmdefault}{put} % Utopia

\ifxetex
\else
\renewcommand*{\rmdefault}{bch} % Charter
\renewcommand*{\ttdefault}{cmtt} % Computer Modern Typewriter
%\renewcommand*{\rmdefault}{phv} % Helvetica
%\renewcommand*{\rmdefault}{iwona} % Avant Garde
\fi

\setlength{\parindent}{0em}  % indentation of paragraph

\setlength{\headheight}{14.5pt}
\pagestyle{fancy}
\fancyfoot[ER,OL]{\thepage}%Page no. in the left on
                                %odd pages and on right on even pages
\fancyfoot[OC,EC]{\sffamily }
\renewcommand{\headrulewidth}{0.1pt}
\renewcommand{\footrulewidth}{0.1pt}
\captionsetup{margin=10pt,font=small,labelfont=bf}


%--- chapter heading

\def\@makechapterhead#1{%
  \vspace*{10\p@}%
  {\parindent \z@ \raggedright %\sffamily
        %{\Large \MakeUppercase{\@chapapp} \space \thechapter}
        %\\
        %\hrulefill
        %\par\nobreak
        %\vskip 10\p@
    \interlinepenalty\@M
    \Huge \bfseries 
    \thechapter \space\space #1\par\nobreak
    \vskip 30\p@
  }}

%---chapter heading for \chapter*  
\def\@makeschapterhead#1{%
  \vspace*{10\p@}%
  {\parindent \z@ \raggedright
    \sffamily
    \interlinepenalty\@M
    \Huge \bfseries  
    #1\par\nobreak
    \vskip 30\p@
  }}
  



% %%%%%%%%%%%%% boxit
\def\Beginboxit
   {\par
    \vbox\bgroup
	   \hrule
	   \hbox\bgroup
		  \vrule \kern1.2pt %
		  \vbox\bgroup\kern1.2pt
   }

\def\Endboxit{%
			      \kern1.2pt
		       \egroup
		  \kern1.2pt\vrule
		\egroup
	   \hrule
	 \egroup
   }	

\newenvironment{boxit}{\Beginboxit}{\Endboxit}
\newenvironment{boxit*}{\Beginboxit\hbox to\hsize{}}{\Endboxit}



\allowdisplaybreaks

\makeatletter
\newcounter{elimination@steps}
\newcolumntype{R}[1]{>{\raggedleft\arraybackslash$}p{#1}<{$}}
\def\elimination@num@rights{}
\def\elimination@num@variables{}
\def\elimination@col@width{}
\newenvironment{elimination}[4][0]
{
    \setcounter{elimination@steps}{0}
    \def\elimination@num@rights{#1}
    \def\elimination@num@variables{#2}
    \def\elimination@col@width{#3}
    \renewcommand{\arraystretch}{#4}
    \start@align\@ne\st@rredtrue\m@ne
}
{
    \endalign
    \ignorespacesafterend
}
\newcommand{\eliminationstep}[2]
{
    \ifnum\value{elimination@steps}>0\leadsto\quad\fi
    \left[
        \ifnum\elimination@num@rights>0
            \begin{array}
            {@{}*{\elimination@num@variables}{R{\elimination@col@width}}
            |@{}*{\elimination@num@rights}{R{\elimination@col@width}}}
        \else
            \begin{array}
            {@{}*{\elimination@num@variables}{R{\elimination@col@width}}}
        \fi
            #1
        \end{array}
    \right]
    & 
    \begin{array}{l}
        #2
    \end{array}
    &%                                    moved second & here
    \addtocounter{elimination@steps}{1}
}
\makeatother

%% Fast macro for column vectors
\makeatletter  
\def\colvec#1{\expandafter\colvec@i#1,,,,,,,,,\@nil}
\def\colvec@i#1,#2,#3,#4,#5,#6,#7,#8,#9\@nil{% 
  \ifx$#2$ \begin{bmatrix}#1\end{bmatrix} \else
    \ifx$#3$ \begin{bmatrix}#1\\#2\end{bmatrix} \else
      \ifx$#4$ \begin{bmatrix}#1\\#2\\#3\end{bmatrix}\else
        \ifx$#5$ \begin{bmatrix}#1\\#2\\#3\\#4\end{bmatrix}\else
          \ifx$#6$ \begin{bmatrix}#1\\#2\\#3\\#4\\#5\end{bmatrix}\else
            \ifx$#7$ \begin{bmatrix}#1\\#2\\#3\\#4\\#5\\#6\end{bmatrix}\else
              \ifx$#8$ \begin{bmatrix}#1\\#2\\#3\\#4\\#5\\#6\\#7\end{bmatrix}\else
                 \PackageError{Column Vector}{The vector you tried to write is too big, use bmatrix instead}{Try using the bmatrix environment}
              \fi
            \fi
          \fi
        \fi
      \fi
    \fi
  \fi 
}  
\makeatother

\robustify{\colvec}

%%% Local Variables: 
%%% mode: latex
%%% TeX-master: "notes"
%%% End: 
 % various packages needed for maths etc.
% quick way of adding a figure
\newcommand{\fig}[3]{
 \begin{center}
 \scalebox{#3}{\includegraphics[#2]{#1}}
 \end{center}
}

%\newcommand*{\point}[1]{\vec{\mkern0mu#1}}
\newcommand{\ci}[0]{\perp\!\!\!\!\!\perp} % conditional independence
\newcommand{\point}[1]{{#1}} % points 
\renewcommand{\vec}[1]{{\boldsymbol{{#1}}}} % vector
\newcommand{\mat}[1]{{\boldsymbol{{#1}}}} % matrix
\newcommand{\R}[0]{\mathds{R}} % real numbers
\newcommand{\Z}[0]{\mathds{Z}} % integers
\newcommand{\N}[0]{\mathds{N}} % natural numbers
\newcommand{\nat}[0]{\mathds{N}} % natural numbers
\newcommand{\Q}[0]{\mathds{Q}} % rational numbers
\ifxetex
\newcommand{\C}[0]{\mathds{C}} % complex numbers
\else
\newcommand{\C}[0]{\mathds{C}} % complex numbers
\fi
\newcommand{\tr}[0]{\text{tr}} % trace
\renewcommand{\d}[0]{\mathrm{d}} % total derivative
\newcommand{\inv}{^{-1}} % inverse
\newcommand{\id}{\mathrm{id}} % identity mapping
\renewcommand{\dim}{\mathrm{dim}} % dimension
\newcommand{\rank}[0]{\mathrm{rk}} % rank
\newcommand{\determ}[1]{\mathrm{det}(#1)} % determinant
\newcommand{\scp}[2]{\langle #1 , #2 \rangle}
\newcommand{\kernel}[0]{\mathrm{ker}} % kernel/nullspace
\newcommand{\img}[0]{\mathrm{Im}} % image
\newcommand{\idx}[1]{{(#1)}}
\DeclareMathOperator*{\diag}{diag}
\newcommand{\E}{\mathds{E}} % expectation
\newcommand{\var}{\mathds{V}} % variance
\newcommand{\gauss}[2]{\mathcal{N}\big(#1,\,#2\big)} % gaussian distribution N(.,.)
\newcommand{\gaussx}[3]{\mathcal{N}\big(#1\,|\,#2,\,#3\big)} % gaussian distribution N(.|.,.)
\newcommand{\gaussBig}[2]{\mathcal{N}\left(#1,\,#2\right)} % see above, but with brackets that adjust to the height of the arguments
\newcommand{\gaussxBig}[3]{\mathcal{N}\left(#1\,|\,#2,\,#3\right)} % see above, but with brackets that adjust to the height of the arguments
\DeclareMathOperator{\cov}{Cov} % covariance (matrix) 
\ifxetex
\renewcommand{\T}[0]{^\top} % transpose
\else
\newcommand{\T}[0]{^\top}
\fi
% matrix determinant
\newcommand{\matdet}[1]{
\left|
\begin{matrix}
#1
\end{matrix}
\right|
}



%%% various color definitions
\definecolor{darkgreen}{rgb}{0,0.6,0}

\newcommand{\blue}[1]{{\color{blue}#1}}
\newcommand{\red}[1]{{\color{red}#1}}
\newcommand{\green}[1]{{\color{darkgreen}#1}}
\newcommand{\orange}[1]{{\color{orange}#1}}
\newcommand{\magenta}[1]{{\color{magenta}#1}}
\newcommand{\cyan}[1]{{\color{cyan}#1}}


% redefine emph
\renewcommand{\emph}[1]{\blue{\bf{#1}}}

% place a colored box around a character
\gdef\colchar#1#2{%
  \tikz[baseline]{%
  \node[anchor=base,inner sep=2pt,outer sep=0pt,fill = #2!20] {#1};
    }%
}%
 % short-hand notation and macros


%%%%%%%%%%%%%%%%%%%%%%%%%%%%

\begin{document}
% front page
% Last modification: Jiahao
\begin{titlepage}

\newcommand{\HRule}{\rule{\linewidth}{0.5mm}} % Defines a new command for the horizontal lines, change thickness here


%----------------------------------------------------------------------------------------


\begin{center} % Center remainder of the page

%----------------------------------------------------------------------------------------
%	HEADING SECTIONS
%----------------------------------------------------------------------------------------
\textsc{\LARGE \reporttype}\\[1.5cm] 
\textsc{\Large Imperial College London}\\[0.5cm] 
\textsc{\large Department of Computing}\\[0.5cm] 
%----------------------------------------------------------------------------------------
%	TITLE SECTION
%----------------------------------------------------------------------------------------

\HRule \\[0.4cm]
{ \huge \bfseries \reporttitle}\\ % Title of your document
\HRule \\[1.5cm]
Coursework 2
\end{center}
%----------------------------------------------------------------------------------------
%	AUTHOR SECTION
%----------------------------------------------------------------------------------------

%\begin{minipage}{0.4\hsize}
\begin{flushleft} \large
\textit{Author:}\\
\reportauthor~(CID: \cid) % Your name
\end{flushleft}
\vspace{2cm}
\makeatletter
Date: \@date 

\vfill % Fill the rest of the page with whitespace



\makeatother


\end{titlepage}




%%%%%%%%%%%%%%%%%%%%%%%%%%%% Main document
\section{Part 1}


\subsection{Q.1}


\begin{enumerate}[a)]
\item
To show that $\vec{\Delta} x_k$ is descent direction at $x_k$, we need to show that:
\begin{align}
\nabla f(x_k)\T \vec{\Delta} x_k < 0 \\
\end{align}
Since:
\begin{align}
\nabla^2 f(x_k) \vec{\Delta} x_k = - \nabla f(x_k)\\
\end{align}
We can get:
\begin{align}
\nabla f(x_k)\T \vec{\Delta} x_k = - \nabla f(x_k)\T  \frac{\nabla f(x_k)}{\nabla^2 f(x_k)}\\
= - \frac{\nabla f(x_k)\T \nabla f(x_k)}{\nabla^2 f(x_k)}
\end{align}
We can see that this is less than zero since the numerator is positive definite and the same for denominator:
\begin{align}
\nabla^2 f(x_k) \succeq m\vec{I}
\end{align}



\item
We can set tolerane to $1.0e^{-08}$ and say that:
\begin{align}
|f(x_{k+1}) - f(x_k)| < tor\\
||\nabla f(x_k)||_2 < tor\\
||x_{k+1} - x_k||_2 < tor
\end{align}
This is to check that the First Order Necessary Condition is satisfied.

\item
Yes, since the function is strongly convex, but the condition is that the initial point $x_0$ has to be close enough to the optimal point.

\item 
First say that:
\begin{align}
x_{k+1} = x_k + t_k \vec{\Delta}x_k
\end{align}
Then 
\begin{align}
f(x_{k+1}) = f(x_k + t_k \vec{\Delta}x_k)
\end{align}
Now use Taylor expansion to expand the above function into second order:
\begin{align}
f(x_k + t_k \vec{\Delta}x_k) \approx f(x_k) + t_k\left\langle\nabla f(x_k), \vec{\Delta}x_k\right\rangle + \frac{1}{2}\nabla^2 f(x_k)||t_k||_2^2||\vec{\Delta}x_k||_2^2\\
\vec{\Delta}x_k = - {\nabla^2 f(x_k)}^{-1}  \nabla f(x_k)\\
\left\langle\nabla f(x_k), \vec{\Delta}x_k\right\rangle = -{\nabla^2 f(x_k)}^{-1} {\nabla f(x_k)}\T \nabla f(x_k)
\end{align}
We need to show that:
\begin{align}
f(x_k) + t_k\langle \nabla f(x_k), \vec{\Delta}x_k\rangle + \frac{1}{2}\nabla^2 f(x_k)||t_k||_2^2||\vec{\Delta}x_k||_2^2 \leq f(x_k) + \alpha t_k \langle\nabla f(x_k), \vec{\Delta}x_k\rangle\\
t_k\langle\nabla f(x_k), \vec{\Delta}x_k)\rangle + \frac{1}{2}\nabla^2 f(x_k)||t_k||_2^2||\vec{\Delta}x_k||_2^2 \leq   \alpha t_k \langle\nabla f(x_k), \vec{\Delta}x_k\rangle\\
\frac{1}{2}\nabla^2 f(x_k)||t_k||_2^2||\vec{\Delta}x_k||_2^2 \leq   (\alpha - 1) t_k \langle\nabla f(x_k), \vec{\Delta}x_k\rangle\\
\frac{1}{2}\nabla^2 f(x_k)t_k||\vec{\Delta}x_k||_2^2 \leq   (\alpha - 1)\langle\nabla f(x_k), \vec{\Delta}x_k\rangle
\end{align}
Since we have condition on $\alpha$:
\begin{align}
0 < \alpha < 0.5\\
-1 < \alpha - 1 < -0.5
\end{align}
Since $\langle\nabla f(x_k), \vec{\Delta}x_k)\rangle < 0$:
\begin{align}
-\langle\nabla f(x_k), \vec{\Delta}x_k)\rangle \geq (\alpha - 1)\langle\nabla f(x_k), \vec{\Delta}x_k)\rangle \geq -0.5\langle\nabla f(x_k), \vec{\Delta}x_k\rangle
\end{align}
So we need to show that:
\begin{align}
\frac{1}{2}\nabla^2 f(x_k)t_k||\vec{\Delta}x_k||_2^2 \leq -\langle\nabla f(x_k), \vec{\Delta}x_k\rangle\\
\nabla^2 f(x_k)t_k||\vec{\Delta}x_k||_2^2 \leq - 2\langle\nabla f(x_k), \vec{\Delta}x_k\rangle\\
t_k \leq - 2\frac{\langle\nabla f(x_k), \vec{\Delta}x_k)\rangle}{\nabla^2 f(x_k)||\vec{\Delta}x_k||_2^2}
\end{align}
Since we have condition on $\nabla^2 f(x_k)$:
\begin{align}
\vec{mI} \preceq   \nabla^2 f(x_k) \preceq \vec{MI}\\
\frac{\langle\nabla f(x_k), \vec{\Delta}x_k)\rangle}{\nabla^2 f(x_k)||\vec{\Delta}x_k||_2^2} \geq \frac{\langle\nabla f(x_k), \vec{\Delta}x_k)\rangle}{\vec{M}||\vec{\Delta}x_k||_2^2}\\
- \frac{\langle\nabla f(x_k), \vec{\Delta}x_k)\rangle}{\nabla^2 f(x_k)||\vec{\Delta}x_k||_2^2} \leq - \frac{\langle\nabla f(x_k), \vec{\Delta}x_k)\rangle}{\vec{M}||\vec{\Delta}x_k||_2^2}\\
\end{align}
We got:
\begin{align}
t_k \leq - 2\frac{\langle\nabla f(x_k), \vec{\Delta}x_k)\rangle}{M||\vec{\Delta}x_k||_2^2}
\end{align}
From the condition we know:
\begin{align}
t_k \leq - \frac{\langle\nabla f(x_k), \vec{\Delta}x_k)\rangle}{M||\vec{\Delta}x_k||_2^2}
\end{align}
Since $t_k$ is positive, so we know the final equation we shown is true.\\

\item
Assume $t_0$ is the initial step size, according to the condition given for Hessian of $f(x_k)$
\begin{align}
\vec{mI} \preceq   \nabla^2 f(x_k) \preceq \vec{MI}\\
\end{align}
We can deduce:
\begin{align}
\vec{mI} \preceq   \nabla^2 f(x_k) \preceq \vec{MI}\\
\vec{mI} \preceq  \frac{\nabla f(x_k) }{x_k} \preceq \vec{MI}\\
\vec{m} \leq  \frac{\left\langle\nabla f(x_k),  x_k\right\rangle }{||x_k||^2} \leq \vec{M}\\
\frac{\vec{m} }{\vec{M}} \leq  \frac{\left\langle\nabla f(x_k),  x_k\right\rangle }{\vec{M}||x_k||^2} \leq 1\\
\frac{\vec{m} }{\vec{M}} \geq  - \frac{\left\langle\nabla f(x_k),  x_k\right\rangle }{\vec{M}||x_k||^2} \geq 1\\
\end{align}
According to the upper bound for $t_k$ when backtracking stop, we got that when
\begin{align}
t_k \leq \frac{\vec{m} }{\vec{M}}
\end{align}
backtracking stops, which is equivalent to
\begin{align}
\beta^n t_0 \leq \frac{\vec{m} }{\vec{M}}
\end{align}
which the $n_0$ is maximum number of iteration for backtracking, and it could be expressed as:
\begin{align}
n_0 = min \quad n \quad st. \quad n \leq \log_\beta \frac{\vec{m}}{\vec{M}t_0} &= \log_\beta \vec{m} - \log_\beta \vec{M} - \log_\beta t_0
\end{align}
\end{enumerate}

\section{Part 2}
\subsection{Q.2}
\begin{enumerate}[a)]
\item
KKT conditions:
\begin{align}
min \quad z_1^2 + (x_2 + 1)^2\\
g(x^*) = exp(x_1^*) - x_2^* \leq 0\\
\mu^* \geq 0\\
2x_1^* + \mu^* exp(x_1^*) = 0\\
2(x_2^* + 1) + \mu^* * (-1) = 0\\
\mu^* (exp(x_1^*) - x_2^*) = 0
\end{align}
if $\mu^* = 0$ :
\begin{align}
2 x_1^* = 0\\
x_1^* = 0\\
2(x_2^* + 1) = 0\\
x_2^* = -1\\
g(x^*) = exp(0) - (-1) = 2 > 0\\
\end{align}
This is contradicting our condition, so $\mu^* > 0$ :
\begin{align}
exp(x_1^*) - x_2^* = 0\\
exp(x_1^*) =  x_2^*\\
2(x_2^* + 1) + \mu^* (-1) = 0\\
2(exp(x_1^*) + 1) = \mu^*\\
2x_1^* + \mu^* * exp(x_1^*) = 0\\
2x_1^* + 2(exp(x_1^*) + 1) * exp(x_1^*) = 0
\end{align}
Since $exp(x_1^*) > 0$:
\begin{align}
2(exp(x_1^*) + 1) * exp(x_1^*) > 0\\
\end{align}
combine with
\begin{align}
2x_1^* + 2(exp(x_1^*) + 1) * exp(x_1^*) = 0
\end{align}
we got
\begin{align}
2x_1^* < 0\\
x_1^* < 0\\
0 < exp(x_1^*) < 1\\
1 < exp(x_1^*) + 1 < 2\\
0 < (exp(x_1^*) + 1) * exp(x_1^*) < 2\\
\end{align}                                               
combine with the above equation agian:
\begin{align}
x_1^* + (exp(x_1^*) + 1) * exp(x_1^*) = 0\\
-2 < x_1^* < 0
\end{align}


\item 
KKT condition:
\begin{align}
min \quad c\T x + 8\\
g(x^*) = \frac{1}{2} ||x^*||^2 -1 \leq 0\\
\mu^* \geq 0\\
c + \mu^* x^* = 0\\
\mu^* (\frac{1}{2} ||x^*||^2 -1) = 0
\end{align}
We can see that if $\mu^* = 0$, $c=0$ as well, which contradicts our given condition $c \neq 0$, so we get $\mu^* > 0$
\begin{align}
\frac{1}{2} ||x^*||^2 -1 = 0\\
||x^*||^2 = 2\\
||a \vec{1}||^2 = 2\\
n|a|^2 = 2\\
a = \pm \sqrt{\frac{2}{n}}\\
c\T x + 8 = 4\\
c\T x = -4\\
n(\vec{c} a) = -4\\
\vec{c} = \frac{-4}{an}\\
c = \mp \frac{4}{\sqrt{2n}}
\end{align}
Please noted that here c and a have inverse signs.

\item 
KKT Condition For original problem:
\begin{align}
min \quad f(x)\\
h(x^*) = 0\\
\nabla f(x^*) + \lambda^* \nabla h(x^*) = 0
\end{align}
KKT Condition For In-Eq problem:
\begin{align}
g(x^*) = \frac{1}{2}||h(x^*)||^2 \leq 0\\
\mu^* \geq 0\\
\nabla f(x^*) + \mu^* h(x^*) = 0\\
\frac{1}{2} \mu^* ||h(x^*)||^2 = 0
\end{align}
If $\mu^* = 0$:
\begin{align}
\nabla f(x^*) = 0\\
h(x^*) may not be 0
\end{align}
So $\mu^* > 0$:
\begin{align}
||h(x^*)||^2 = 0\\
\nabla g(x^*) = ||h(x^*)|| = 0\\
h(x^*) = 0
\end{align}
So $h(x^*)$ is not linearly independent, and $x^*$ is not a regular point, so KKT theorem can not be applied on In-Eq problem.


\end{enumerate}

\end{document}
%%% Local Variables: 
%%% mode: latex
%%% TeX-master: t
%%% End: 
