\documentclass[12pt,twoside]{article}

\newcommand{\reporttitle}{477 - Computational Optimisation}
\newcommand{\reportauthor}{Jiahao Lin}
\newcommand{\reporttype}{Coursework 2}
\newcommand{\cid}{00837321}

% include files that load packages and define macros
\input{includes} % various packages needed for maths etc.
\input{notation} % short-hand notation and macros


%%%%%%%%%%%%%%%%%%%%%%%%%%%%

\begin{document}
% front page
\input{titlepage}


%%%%%%%%%%%%%%%%%%%%%%%%%%%% Main document
\section{Part 1}


\subsection{Q.1}


\begin{enumerate}[a)]
\item
To show that $\vec{\Delta} x_k$ is descent direction at $x_k$, we need to show that:
\begin{align}
\nabla f(x_k)\T \vec{\Delta} x_k < 0 \\
\end{align}
Since:
\begin{align}
\nabla^2 f(x_k) \vec{\Delta} x_k = - \nabla f(x_k)\\
\end{align}
We can get:
\begin{align}
\nabla f(x_k)\T \vec{\Delta} x_k = - \nabla f(x_k)\T  \frac{\nabla f(x_k)}{\nabla^2 f(x_k)}\\
= - \frac{\nabla f(x_k)\T \nabla f(x_k)}{\nabla^2 f(x_k)}
\end{align}
We can see that this is less than zero since the numerator is positive definite and the same for denominator:
\begin{align}
\nabla^2 f(x_k) \succeq m\vec{I}
\end{align}



\item
We can set tolerane to $e^{-08}$ and say that:
\begin{align}
|f(x_{k+1}) - f(x_k)| < tor\\
||\nabla f(x_k)||_2 < tor\\
||x_{k+1} - x_k||_2 < tor
\end{align}
This is to check that the First Order Necessary Condition is satisfied.

\item
Yes, since the function is strongly convex, but the condition is that the initial point $x_0$ has to be close enough to the optimal point.

\item 
First say that:
\begin{align}
x_{k+1} = x_k + t_k \vec{\Delta}x_k
\end{align}
Then 
\begin{align}
f(x_{k+1}) = f(x_k + t_k \vec{\Delta}x_k)
\end{align}
Now use Taylor expansion to expand the above function into second order:
\begin{align}
f(x_k + t_k \vec{\Delta}x_k) \approx f(x_k) + t_k\left\langle\nabla f(x_k), \vec{\Delta}x_k\right\rangle + \frac{1}{2}\nabla^2 f(x_k)||t_k||_2^2||\vec{\Delta}x_k||_2^2\\
\vec{\Delta}x_k = - \frac{\nabla f(x_k)}{\nabla^2 f(x_k)}\\
\left\langle\nabla f(x_k), \vec{\Delta}x_k\right\rangle = - \frac{||\nabla f(x_k)||_2^2}{\nabla^2 f(x_k)}
\end{align}
We need to show that:
\begin{align}
f(x_k) + t_k\langle \nabla f(x_k), \vec{\Delta}x_k\rangle + \frac{1}{2}\nabla^2 f(x_k)||t_k||_2^2||\vec{\Delta}x_k||_2^2 \leq f(x_k) + \alpha t_k \langle\nabla f(x_k), \vec{\Delta}x_k)\rangle\\
t_k\langle\nabla f(x_k), \vec{\Delta}x_k)\rangle + \frac{1}{2}\nabla^2 f(x_k)||t_k||_2^2||\vec{\Delta}x_k||_2^2 \leq   \alpha t_k \langle\nabla f(x_k), \vec{\Delta}x_k)\rangle
\end{align}
Now let's look at first term fisrt:
\begin{align}
t_k\langle\nabla f(x_k), \vec{\Delta}x_k)\rangle
\end{align}
Since we have the condition on $t_k$:
\begin{align}
t \leq - \frac{\langle\nabla f(x_k), \vec{\Delta}x_k)\rangle}{\vec{M} ||\nabla x_k||_2^2}\\
0 > {\langle\nabla f(x_k), \vec{\Delta}x_k)\rangle}t_k > -{\langle\nabla f(x_k), \vec{\Delta}x_k)\rangle}\T \frac{\langle\nabla f(x_k), \vec{\Delta}x_k)\rangle}{\vec{M} ||\nabla x_k||_2^2}\\
0 > {\langle\nabla f(x_k), \vec{\Delta}x_k)\rangle}t_k > -\frac{||\nabla f(x_k)||^2 ||\vec{\Delta}x_k)||^2}{\vec{M} ||\nabla x_k||_2^2}\\
0 > {\langle\nabla f(x_k), \vec{\Delta}x_k)\rangle}t_k > -\frac{||\nabla f(x_k)||^2}{\vec{M}}
\end{align}
Now let's take a look at second term:
\begin{align}
\frac{1}{2}\nabla^2 f(x_k)||t_k||_2^2||\vec{\Delta}x_k||_2^2
\end{align}
Again substitute in condition for $t$:
\begin{align}
t \leq - \frac{\langle\nabla f(x_k), \vec{\Delta}x_k)\rangle}{\vec{M} ||\nabla x_k||_2^2}\\
||t||_2^2 \leq \frac{||\nabla f(x_k)||^2 ||\vec{\Delta}x_k)||^2}{\vec{M}\T \vec{M} ||\nabla x_k||_2^4}\\
||t||_2^2 \leq \frac{||\nabla f(x_k)||^2}{\vec{M}\T \vec{M} ||\nabla x_k||_2^2}\\
||t||_2^2 ||\vec{\Delta}x_k||_2^2 \leq \frac{||\nabla f(x_k)||^2}{\vec{M}\T \vec{M}}\\
\end{align}
Since we have the upper bound for $\nabla^2 f(x_k)$:
\begin{align}
\nabla^2 f(x_k) \preceq \vec{MI}\\
||t||_2^2 ||\vec{\Delta}x_k||_2^2 \nabla^2 f(x_k) \leq \frac{\nabla^2 f(x_k) ||\nabla f(x_k)||^2}{\vec{M}\T \vec{M}}\\
\end{align}



\item 
\end{enumerate}

\section{Part 2}
\subsection{Q.2}
\begin{enumerate}[a)]
\item
KKT conditions:
\begin{align}
min \quad z_1^2 + (x_2 + 1)^2\\
g(x^*) = exp(x_1^*) - x_2^* \leq 0\\
\mu^* \geq 0\\
2x_1^* + \mu^* exp(x_1^*) = 0\\
2(x_2^* + 1) + \mu^* * (-1) = 0\\
\mu^* (exp(x_1^*) - x_2^*) = 0
\end{align}
if $\mu^* = 0$ :
\begin{align}
2 x_1^* = 0\\
x_1^* = 0\\
2(x_2^* + 1) = 0\\
x_2^* = -1\\
g(x^*) = exp(0) - (-1) = 2 > 0\\
\end{align}
This is contradicting our condition, so $\mu^* > 0$ :
\begin{align}
exp(x_1^*) - x_2^* = 0\\
exp(x_1^*) =  x_2^*\\
2(x_2^* + 1) + \mu^* (-1) = 0\\
2(exp(x_1^*) + 1) = \mu^*\\
2x_1^* + \mu^* * exp(x_1^*) = 0\\
2x_1^* + 2(exp(x_1^*) + 1) * exp(x_1^*) = 0
\end{align}
Since $exp(x_1^*) > 0$:
\begin{align}
2(exp(x_1^*) + 1) * exp(x_1^*) > 0\\
\end{align}
combine with
\begin{align}
2x_1^* + 2(exp(x_1^*) + 1) * exp(x_1^*) = 0
\end{align}
we got
\begin{align}
2x_1^* < 0\\
x_1^* < 0\\
0 < exp(x_1^*) < 1\\
1 < exp(x_1^*) + 1 < 2\\
0 < (exp(x_1^*) + 1) * exp(x_1^*) < 2\\
\end{align}                                               
combine with the above equation agian:
\begin{align}
x_1^* + (exp(x_1^*) + 1) * exp(x_1^*) = 0\\
-1 < x_1^* < 0
\end{align}


\item 
KKT condition:
\begin{align}
min \quad c\T x + 8\\
g(x^*) = \frac{1}{2} ||x^*||^2 -1 \leq 0\\
\mu^* \geq 0\\
c + \mu^* x^* = 0\\
\mu^* (\frac{1}{2} ||x^*||^2 -1) = 0
\end{align}
We can see that if $\mu^* = 0$, $c=0$ as well, which contradicts our given condition $c \neq 0$, so we get $\mu^* > 0$
\begin{align}
\frac{1}{2} ||x^*||^2 -1 = 0\\
||x^*||^2 = 2\\
||a \vec{1}||^2 = 2\\
n|a|^2 = 2\\
a = \pm \sqrt{\frac{2}{n}}\\
c\T x + 8 = 4\\
c\T x = -4\\
n(\vec{c} a) = -4\\
\vec{c} = \frac{-4}{an}\\
c = \mp \frac{4}{\sqrt{2n}}
\end{align}
Please noted that here c and a have inverse signs.

\item 
KKT Condition For original problem:
\begin{align}
min \quad f(x)\\
h(x^*) = 0\\
\nabla f(x^*) + \lambda^* \nabla h(x^*) = 0
\end{align}
KKT Condition For In-Eq problem:
\begin{align}
g(x^*) = \frac{1}{2}||h(x^*)||^2 \leq 0\\
\mu^* \geq 0\\
\nabla f(x^*) + \mu^* h(x^*) = 0\\
\frac{1}{2} \mu^* ||h(x^*)||^2 = 0
\end{align}
If $\mu^* = 0$:
\begin{align}
\nabla f(x^*) = 0
h(x^*) may not be 0
\end{align}
So $\mu^* > 0$:
\begin{align}
||h(x^*)||^2 = 0
\nabla g(x^*) = ||h(x^*)|| = 0
h(x^*) = 0
\end{align}
So $h(x^*)$ is not linearly independent, and $x^*$ is not a regular point, so KKT theorem can not be applied on In-Eq problem.


\end{enumerate}

\end{document}
%%% Local Variables: 
%%% mode: latex
%%% TeX-master: t
%%% End: 
